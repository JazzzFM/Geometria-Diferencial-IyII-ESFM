\section{Cálculo Variacional}
\subsection{Ecuación de Euler}
\begin{proposition}
  Sea $G$ un abierto en $\mathbb{R}^2$ y $f:G \to \mathbb{R}$ una función
  continua tal que $D_1f$ existe y es continua en $G$. Suponga que $[\alpha,
  \beta] \times [a,b] \subset G$. Para $x \in [\alpha,\beta]$, considere 
  \begin{align*}
    f_x:[a,b] &\to \mathbb{R} \\
    y &\mapsto f(x,y)
  \end{align*}
  Y sea 
  \begin{align*}
    F: [\alpha, \beta] &\to \mathbb{R} \\
    x &\mapsto \int^b_{a} f_x
  \end{align*}
  Entonces $\forall \, x \in [\alpha, \beta]$, $F'(x) $ existe y es igual 
  \[
    \int_a^b (D_1f)_x
  \]
  Donde 
  \begin{align*}
    (D_1f)_x : [a,b] &\to \mathbb{R} \\
    y &\mapsto D_1(x,y)
  \end{align*}
\end{proposition}
\begin{proof}
 Claramente $f_x$ es continua. En consecuencia $\int_a^b f_x$ existe, y por ello
 $F$ está definida. Considere un elemento $(x,y)$ de $[\alpha, \beta] \times
 [a,b]$, si definimos
 \begin{align*}
   f^y: [\alpha,\beta] &\to \mathbb{R} \\
   x &\mapsto f(x,y)
 \end{align*}
 Se tiene así que $(f^y)'(x) = D_1f(x,y)$.

 Sean $\hat{x} \in [\alpha, \beta]$, $\hat{h} \neq 0$ y $\hat{x} + \hat{h} \in
 [\alpha, \beta]$. Entonces
 \[
   F(\hat{x} + \hat{h}) - F(\hat{x}) = \int_a^b f_{\hat{x} + \hat{h}} - \int_a^b
 f_{\hat{x}} = \int_a^b f_{\hat{x} + \hat{h}} - f_{\hat{x}} 
 \]
 Por otra parte, del teorema del valor medio existe $c \in (\hat{x},\hat{x} +
 \hat{h})$
 \[
   (f^y)'(c) = \frac{f^y(\hat{x} + \hat{h}) - f^y(\hat{x})}{\hat{h}}
 \]
 $c$ entonces  se puede escribir como $c = \hat{x} + \theta\hat{h}$ con $\theta
 \in (0,1)$. Además $\theta$ depende de $y$
  De ésta manera
  \begin{align*}
    f(\hat{x} + \hat{h}, y) - f(\hat{x}, y) &= \hat{h}(f^y)'(c) \\ 
    &= \hat{h}D_1f(c,y) \\ 
    &= \hat{h}D_1f(\hat{x} + \theta \hat{h},y) 
  \end{align*}
  Como $D_1f$ es continua en un compacto, dado $\epsilon > 0$
  $\exists \, \delta > 0$ tal que 
  \[
    |D_1f(c, y) - D_1(\hat{x}, y)| < \frac{\epsilon}{ b-a}
  \]
  si
  \[
    |\hat{h}| < \epsilon
  \]
  Considere la función 
  \begin{align*}
    g: [a,b] &\to \mathbb{R}^2 \\
    y &\mapsto (c,y)
  \end{align*}
  De lo cual se tiene
  \[
    f_{\hat{x} + \hat{h}} - f_{\hat{x}} = \hat{h} f \circ g
  \]
  Así
  \begin{align*}
    \frac{F( \hat{x} + \hat{h}) - F(\hat{x})}{\hat{h}} &= \int_a^b (D_1
    f)_{\hat{x}} \\ 
    \big| \frac{F( \hat{x} + \hat{h}) - F(\hat{x})}{\hat{h}} - \int_a^b (D_1
    f)_{\hat{x}} &\leq \int_a^b \big| D_1f\circ g - (D_1f)_{\hat{x}} \big|
  \end{align*}
  Así se puede finalmente concluir 
  \[
    \lim_{\hat{h} \to 0} \frac{F(\hat{x} + \hat{h}) - F(\hat{x})}{\hat{h}}
  \]
  existe y es igual a $\int_a^b (D_1 f)_{\hat{x}}$, y dado el hecho que $D_1f$
  es uniformemente continua, la derivada existe y es continua.
\end{proof}
\begin{lemma}
  Sea $f: [a, b] \to \mathbb{R}$ una función continua, tal que
  \[
    \int_a^b f \eta = 0
  \]
 para toda función $\eta: [a,b] \to \mathbb{R}$ continua que satisface
 $\eta(a) = \eta(b) = 0$. Entonces $f(x) = 0 \quad \forall \, x \in [a, b]$

\end{lemma}
\begin{proof}
  Suponga que existe $c \in [a,b]$ tal que $f(c) \neq 0$. Suponga que $f(c)>0$.
  Como $f$ es continua, para $\epsilon =f(c)$ existe $\delta > 0 $ tal que 
  \[
    | f(x) - f(c) | < \epsilon \mbox{ si } | x-c| < \delta%
    \mbox{ y } x \in [a, b]
  \]
  Así si $x \in [a, b] \cap ( c- \delta, c + \delta)$, entonces $-\epsilon <
  f(x) - f(c)$. Con lo que $0 < f(x)$. Existen entonces $a', b'$; $a'<b'$. tales
  que $(a',b') \subset (a,b) \cap (c-\delta, c+ \delta)$. Considere entonces la
  función $\eta: [a,b] \to \mathbb{R}$ definida en la siguiente forma
  \[
    \eta(x) =%
    \begin{cases}
      (b'-x)(x-a') &\mbox{ si } x \in [a',b'] \\ 
      0 &\mbox{ si } x \notin [a',b']  
    \end{cases}
  \]
  $\eta$ es continua. Además que $\eta(a)= \eta (b) = 0$. En consecuencia
  \[
    \int_a^b = 0
  \]
  Sin embargo $f\eta$ es estrictamente mayor que 0 en el abierto $(a',b')$ y
  cero en el resto de su dominio, finalmente
  \[
    \int_a^b f\eta > 0
  \]
  Lo cual es una contradicción

\end{proof}

Sea $F: G \subset \mathbb{R}^3 \to \mathbb{R}$ una función diferenciable en el
abierto $G$. Suponga que $[a,b] \subset \Pi_1(G)$, siendo
\begin{align*}
  \Pi : \mathbb{R}^3 &\to \mathbb{R} \\
  (x,y,z) &\mapsto x
\end{align*}
Sean $c, d \in \mathbb{R}$. $C$ el conjunto
\[
  \{f: [a,b] \to \mathbb{R} \, | \, f \mbox{ es de clase} C^2; f(a) = c%
  , f(b) = d, \, (x, f(x), f'(x)) \in G \}
\]
Dado $f \in C$. Sean
\begin{align*}
  g_f : [a, b] &\to G \\
  x &\mapsto (x, f(x), f'(x)) \\
  h_f : F \circ g_f
\end{align*}
Considere finalmente la función
\begin{align*}
  I: C &\to \mathbb{R} \\
  f &\mapsto \int_a^b h_f
\end{align*}
El propósito del cálculo variacional es determinar un elemento $\phi \in C$ tal
que $I(\phi)$ sea máximo o mínimo.
\begin{proposition}[Euler 1977]
 Una condición necesaria para que $u \in C$ sea un extremo de $I$, es que se
 cumpla la siguiente identidad
 \[
   D_2F(x,u(x),u'(x)) - \frac{\diff}{\diff x} [D_3 F(x, u(x), u'(x)]  
 \]
\end{proposition}
\begin{proof}
 Asuma que $u \in C$ es un extremo de $I$.
 Sea 
 \begin{align*}
   \eta : [a,b] &\to \mathbb{R}
 \end{align*}
 una función de clase $C^2$ tal que $\eta(a) = \eta(b) = 0$. Para probar ésta
 proposición, basta probar que:
 \[
   \int_a^b \eta \big[ D_2F(x,u(x),u'(x)) - \frac{\diff}{\diff x}%
 [D_3 F(x, u(x), u'(x)] \, \big]
 \]
 Sea $\hat{\mathbb{R}} = \{ \, \epsilon \in \mathbb{R} \, | \, u + \epsilon\eta
   \in C \}$. Entonces $\{ u + \epsilon\eta \, | \, \epsilon \in
 \hat{\mathbb{R}}\} \in C$.

 Considere las funciones 
 \begin{align*}
   \hat{f}: \hat{\mathbb{R}} &\to C \\
   \epsilon  &\mapsto u + \epsilon\eta \\
   L : [a,b] \times \hat{\mathbb{R}} &\to \mathbb{R}^3 \\
   (x, \epsilon) &\mapsto g_{\hat{f}(\epsilon)}(x) = (x, \hat{f}(x),%
   \hat{f}'(x))
 \end{align*}
 Se tiene así la función:
 para cada $\epsilon \in \hat{\mathbb{R}}$, considere la función
 \begin{align*}
   (F \circ L): [a,b] &\to \mathbb{R} \\
   x &\mapsto (F \circ L)(x, \epsilon)
 \end{align*}
 Considere así mismo la función
 \begin{align*}
 G_{\eta} : \hat{\mathbb{R}} &\to \mathbb{R} \\
 \epsilon &\mapsto I(\hat{f}(\epsilon)) 
 \end{align*}
 Se tiene que $G_{\eta}(\epsilon) = \int_a^b h_{\hat{f}(\epsilon)}$

 Como $F\circ L$ y $(F \circ L)_{\epsilon}$ son continuas, la proposición
 anterior, implica que $G'_{\eta}(\epsilon)$ existe y es igual a $\int_a^b
 D_2(F\circ L)_{\epsilon}$.

 Por otra parte, 
 \[
   D(F\circ L)(x, \epsilon) = DF(L(x, \epsilon)) \circ DL(x, \epsilon)
 \]
 Así 
 \[
   D_2 (F \circ L) = \eta(x)D_2F(L(x, \epsilon)) + \eta'(x)%
 D_3F(L(x,\epsilon)) 
 \]
 Tenemos entonces que 
 \[
   G'_{\eta}(\epsilon) = \int_a^b%
   D_2 (F \circ L) = \eta(x)D_2F(L(x, \epsilon)) + \eta'(x)%
 D_3F(L(x,\epsilon))
 \]
 Utilizando el método de integración por partes se tiene:
 \[
   G_{\eta}'(\epsilon) = \int_a^b \eta(x)\big[D_2(F\circ L(x,\epsilon)) -%
   \frac{\diff}{\diff x} [D_3 (F\circ L(x,\epsilon))] \big] 
 \]
 al ser $\eta(a) = \eta(b) = 0$.
 Como $G_{\eta}(0) = I(u)$.  0 es un extremo de $G_{\eta}$. Así $G_{\eta}' = 0$.
 Entonces
 \[
   \int_a^b \eta \big[ D_2F(x,u(x),u'(x)) - \frac{\diff}{\diff x}%
 [D_3 F(x, u(x), u'(x)] \, \big]
 \]

\end{proof}

\subsection{Geodésicas en la esfera}
Sean $P_1$, $P_2$ dos puntos en la esfera $S^2$. Sea $P: [u,v] \to \mathbb{R}^3$
una curva regular cuya traza está contenida en la esfera, tal que $P(u) = P_1,
P(v) = P_2$.
Utilizando coordenadas esféricas se puede escribir
\begin{align*}
  x(t) &= \cos \theta (t) \cos \phi(t) \\
  y(t) &= \cos \theta (t) \sin \phi(t) \\
  z(t) &= \sin \theta(t) 
\end{align*}

La longitud de la curva es entonces
\[
  L_P = \int_u^v ||P'(t)||
\]

Se tiene entonces que 
\[
  ||P'(t)|| = [\theta'(t)^2 + \phi'(t)^2\sin^2\theta(t)]^{1/2}
\]

Si $\theta$ es inyectiva, podemos escribir
\[
  L_P = \int_u^v \theta'(t) \sqrt{1 + \big[\frac{\phi'(t)}{\theta'(t)}\big]^2%
    \sin^2\theta(t)}
\]
Entonces, utilizando el primer teorema de sustitución
\begin{align*}
L_P = \int_u^v G(t) &= \int_{\theta^{-1}(\alpha)}^{\theta^{-1}(\beta)} \\
                    &=\int_{\alpha}^{\beta} G (\theta^{-1}(x))(\theta^{-1})'(x) \\
                    &= \int_{\alpha}^{\beta} \big[ 1 +%
                      \phi'(\theta^{-1}(x))^2(\theta^{-1})'(x)^2\sin^2x%
                    \big]^{1/2}
\end{align*}
Considere entonces a $f = $
\begin{align*}
  F: \mathbb{R}^3 &\to \mathbb{R} \\
(x,y,z) &\mapsto (1+ x^2\sin^2x)^{1/2}
\end{align*}
Se tiene así la función:
\begin{align*}
  h_f = F \circ g_f: [\alpha, \beta] \to \mathbb{R} 
\end{align*}
Así la ecuación de Euler se reduce en éste caso:
\[
  \frac{\diff}{\diff x} [D_3 (F(x,f(x),f'(x)))] = 0$
\]
Así $\exists C \in \mathbb{R}$ tal que 
\[
  C = D_3 F(x,f(x),f'(x))
\]
Por lo cual 
\[
  \frac{f'(x) \sin^2 x}{\big[ 1 + f'(x)^2 \sin^2x\big]^{1/2}} = C
\]
Lo anterior constituye una condición necesaria para que $P$ sea una geodésica de
$S^2$.

Sea $\Phi = f'$.
Se tiene entonces que 

\[
  \Phi^2 \sin^2x (\sin^2 x - C^2) = C^2
\]
Observe que $|C|<1$

%%%Añadir lo del inicio
Así
\[
  \Phi = \frac{C}{\sin x [\sin^2 x - C^2]^{1/2}}
\]
Reemplazando $\sin^2 x$ se tiene
\begin{align*}
  f' &= \frac{C}{\sin^2x[1-C^2(1+ \cot^2x)]^{1/2}} \\
     &= \frac{C(\cot x)'}{1-C^2 - C^2\cot^2 x]^{1/2}} \\
  f  &= \arccos \frac{C\cot x}{\sqrt{1-C^2}} + C_1  \\
  &= \arccos[ C_2 \cot x] +C_1   
\end{align*}
Como $x=\theta$, se concluye:
\[
  \cos(f(\theta) - C-1) =  C_2 \cot \theta
\]
ó bien
\[
  A\cos f(\theta) + B \sin f(\theta) =  \ctg \theta
\]
Multiplicando ambos miembros por $\sin \theta$, se tiene
\begin{align*}
  A\sin\theta \cos f(\theta) + B \sin \theta \sin f(\theta) &= \cos \theta \\
  A\sin\theta \cos \phi + B \sin\theta\sin\phi &= \cos \theta 
\end{align*}
o bien en coordenadas cartesianas
\[
  Ax + By = z
\]
Es decir que la curva se encuentra en la intersección de una superficie que
contiene al origen

Charles Fox. 

\begin{problem}
  ~\begin{enumerate}
    \item Verifique que si $D_1F = 0$, entonces existe $c \in \mathbb{R}$ tal que
      \[
        F(x,u,u') - u' D_3F(x,u,u') =C
      \]
    \item Verifique que la hélice $f(t) = (\cos t, \sin t, at)$ satisface la
      ecuación de Euler para ser una geodésica del cilindro circular. 

  \end{enumerate}

\end{problem}


Considerando al conjunto $C$ definido como antes, y a la función $I$ de la misma
forma.
Sea $u$ un extremo de $I$
\begin{align*}
  g_u : [a,b] &\to \mathbb{R}^3 \\
  x &\mapsto (x,u(x),u'(x))
\end{align*}
Sea $H = (D_3F)\circ g_u$. Para cada $x \in [a,b]$ se tiene:
\[
  DH(x) = D (D_3F(g_u(x)) \circ D g_u(x)
\]
Así 
\[  
  H'(x) = D_{13}F(g_u(x)) + u'(x)D_{23}F(g_u(x)) + u''(x)D_{33}F(g_h(x))
\]
Como $\frac{\diff}{\diff x} H(x) = H'(x)$.

En consecuencia la ecuación de Euler puede escribirse en la forma:
\[
  -D_2F(x,u(x),u'(x)) + D_{13}F(x,u(x),u'(x)) + u'(x)D_{23}F(x,u(x),u'(x)%
  + u''(x)D_{33}F(x,u(x),u'(x))
\]
Considere una función $F: G \subset \mathbb{R}^{2n +1} \to \mathbb{R}$
diferenciable.
Suponga que $[a,b] \subset \Pi_1(G)$
Sean
\begin{align*}
  C &= \{ f:[a,b] \to \mathbb{R} \, | \, f \in C^2, f(a) = c, f(b)=d \\
    &= (x,f(x), f'(x)) \in G \quad \forall \, x \in [a,b] \} \\
    f \in C \quad ; \quad g_f [a,b] &\to G  \\
    x &\mapsto (x,f(x), f'(x)) \\
    h_f = F \circ g_f : [a,b] &\to \mathbb{R} \\
    x &\mapsto F(x, f(x), f'(x))
\end{align*}
Veamos que $h_f$ es continua. Así podemos considerar la función
\begin{align*}
  I:  C &\to \mathbb{R} \\
  f &\mapsto \int_a^b h_f
\end{align*}
Suponga que $u \in C$ es un extremo de $I$.
Sea $\eta: [a,b] \to \mathbb{R}^n$ una función de clase $C^2$ tal que
$\eta(a)=\eta(b) = 0$ y sea $\hat{\mathbb{R}} = \{ \epsilon \in \, | \, u +
\epsilon\eta \in C \}$. 
Considere las funcióne 
\begin{align*}
  \hat{f} : \hat{\mathbb{R}} &\to C \\
  \epsilon &\mapsto u + \epsilon \eta \\
  G_{\eta}: \hat{\mathbb{R}} &\to \mathbb{R} \\
  \epsilon &\mapsto I(\hat{f}(\epsilon)) = \int_a^bh_{\hat{f}(\epsilon)}
\end{align*}
Sea también:
\begin{align*}
  L : [a,b] \times \hat{\mathbb{R}} &\to \mathbb{R}^{2n +1} \\
  (x, \epsilon)& \mapsto g_{\hat{f}(\epsilon)}(x) \\
  F \circ L : [a,b] \times \hat{\mathbb{R}} &\to \mathbb{R} \\
  (x, \epsilon)& \mapsto F \circ g_{\hat{f}(\epsilon)}(x)
\end{align*}
Dado $\epsilon \in \hat{\mathbb{R}}$, se tiene la función 
\begin{align*}
  h_{\hat{f}(\epsilon)} = (F\circ L)_{\epsilon} : [a,b] &\to \mathbb{R} \\
    x &\mapsto F\circ L (x,\epsilon)
\end{align*}
Se tiene entonces:
\[
  G_{\eta}(\epsilon) = \int_a^b (F \circ L)_{\epsilon}
\]
La proposición 21 implica que
\[
  G'_{\eta}(\epsilon) = \int_a^b [D_2 (F \circ L)]_{\epsilon}
\]
Veamos que 
\[
  [D F(L(x,\epsilon))] =%
  \begin{bmatrix}%
    D_1F(x,\epsilon) \ldots D_{2n +1}%
  \end{bmatrix}
\]
y
\[
  [D L(x, \epsilon)] =%
  \begin{bmatrix}%
    1 & 0 \\
    u_1'(x) + \epsilon\eta_1'(x) & \eta_1'(x) \\
    \vdots & \vdots \\
    u_n'(x) + \epsilon\eta_n'(x) & \eta_n'(x) \\
    u_1''(x) + \epsilon\eta_1''(x) & \eta_1''(x) \\
    \vdots & \vdots \\
    u_n''(x) + \epsilon\eta_n''(x) & \eta_n''(x)
  \end{bmatrix}
\]
Así 
\[
  [D_2(F\circ L)]_{\epsilon} = \eta_1(x)D_2F(L(x,\epsilon)) + \ldots%
  + \eta_n(x)D_{n+1}F(L(x,\epsilon)) +%
  \eta_1'(x)D_{n+2}F(L(x,\epsilon)) +\ldots %
  +\eta_n'D_{2n+1}F(L(x,\epsilon))
\]
Con lo cual, integrando por partes tenemos
\[
  G'_{\eta}(\epsilon) = \int_a^b \sum_{i=1}^n \eta_i[(x)D_{i+1}F(L(x,\epsilon)) -%
  \frac{\diff}{\diff{x}}D_{i+n+1}F(L(x,\epsilon))
\]
En particular $G_{\eta}$ tiene un punto crítico en 0. Así
\[
  0 = G'_{\eta}(\epsilon) = \int_a^b \sum_{i=1}^n \eta_i[(x)D_{i+1}F(L(x,0)) -%
  \frac{\diff}{\diff{x}}D_{i+n+1}F(L(x,0))
\]
Como $\eta$ es una función de clase $C^2$ con $\eta(a)=\eta(b)=0$ arbitraria,
podemos tomar a $\eta^k_k = \eta_k$. $\eta_k^i=0$ para $i \neq k$. Con lo cual
se tiene
\[
  0 = G'_{\eta_k}(\epsilon) = \int_a^b  \eta_k[(x)D_{k+1}F(L(x,0)) -%
  \frac{\diff}{\diff{x}}D_{k+n+1}F(L(x,0))
\]
Así, del lema de la proposición 21, se conluye:
\[
  D_{k+1} F(x, u(x), u'(x)) - \frac{\diff}{\diff x} D_{k+n+1}F(x,u(x),u'(x)) = 0
\]
Con lo cual se muestra lo siguiente:
\begin{proposition}
 Sean $F$ definido como anterior, $C$ el conjunto de funciones anteriores y $u
 \in C$ tal que $u $ es un extremo de $I$, entonces 
\[
  D_{k+1} F(x, u(x), u'(x)) - \frac{\diff}{\diff x} D_{k+n+1}F(x,u(x),u'(x)) = 0
\]
\end{proposition}
Herman  H . Goldstine A History of the calculus of Variations from the 17th
through the 19th century.
\section{Principio de Hamilton}
El principio de Hamilton dice que
para cualquier sistema físico descrito en términos de las $n$ coordenadas
generalizadas cuales definen las posiciones y velocidades de las partículas que
lo comprenden, sobre cualquier conjunto de restricciones; la integral de la
energía cinética menos la energía potencial respecto al tiempo en la trayectoria
que sigue la partícula, es mínima.

Así, tratándose de una partícula cuyo movimiento está descrito por sus tres
componentes, infiere la expresión
\[
  \int_{t_0}^{t_1} T(t,x(t),x'(t))-V(t,x(t),x'(t)) \diff t = \int_{t_0}^{t_1}%
  L(t,x,x') \diff t
\]
Donde a la función $L$ se le denomina función Lagrangiana. Como $L$ es de la
forma en que se deduce la ecuación de Euler de las proposiciones anteriores. 
Así se deduce que la función $x(t)$ debe ser un extremo de
\begin{align*}
  I: C &\to \mathbb{R} \\
  f &\mapsto \int_{t_1}^t h_f
\end{align*}
Tomando $F = L$. Así de la proposición 23 se desprende la exptesión:
\[
  D_{k+1}L(t,x(t),x'(t)) - \frac{\diff}{\diff t} D_{3+k+1} L(t,x(t),x'(t))
\]
Las ecuaciones anteriores son las ecuaciones de Lagrange para la partícula.
En general se escriben:
\[
  \frac{\partial L}{\partial x_k} - \frac{\diff}{\diff t} \frac{\partial%
  L}{\partial x'_k} = 0
\]
\section{La curva braquistocrona} 
...

Si $s(t)$ es la distancia recorrida por la partícula en el tiempo, entonces $v =
s'(t)$. Por el teorema de la función 

...

y considerando a $s$ como función de $x$, $(s(0), s(x_0) = L)$ 

vblsd


