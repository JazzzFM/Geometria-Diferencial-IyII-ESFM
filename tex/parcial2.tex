\section{Evolutas y envolventes}
2-sept-2019

Sea $f: I \to \mathbb{R}^n$ una curva regular. 
$s: I \to \mathbb{R}$ es una función contínua. 

Sean $x,y \in I$ con $x<y$. entonces
\[
  s(y) = \int_a^y||f'|| = \int_a^x||f'|| + \int_x^y||f'|| = s(x) +
  \int_x^y||f'||
\]
Como $||f'||>0$ se tiene que $s(y)>s(x)$.
Sea $J = s(I)$. $J$ es un intervalo cerrado y $s:I\to J$ es una biyeccion
contínua creciente. Así $s^{-1}:J \to I$ es contínua.
Sea $F = f\circ s^{-1}$. $F \in C^n$.
Observe que $(f,F) \in R$ ya que $f = F\circ s$. A $F$ se le denomina la
parametrización natural ó representación paramétrica natural de $[f]$.
\begin{proposition}
  Sea $f: I \to \mathb{R}^n$ una curva regular, $F$ la representación
  paramétrica natural de $[f]$, entonces $F$ es una curva regular y $||F'(x)||
  =1 \quad \forall \, x \in s(I)$.
\end{proposition}
\begin{proof}
  Dado $x \in s(I)$, $x = s(t)$ para algún $t \in I$. Se tiene, dado que $s'(t)
  = ||f'(t)|| \neq 0 \quad \forall t \in I$.
  Se tiene por el teorema de la función inversa que:
  \[
    (s^{-1})'(x) = \frac{1}{s'(u)}
  \]
 En consecuencia $F$ es derivable; además
 \[
   F'(x) = f'(s^{-1} (x)) (s^{-1})'(x) = \frac{f'(u)}{||f'(u)||}
 \]
 Así $||F'(x)||=1$.
\end{proof}
Así para $F$ las ecuaciones de Frenet-Serret son:

\[
  \begin{array}{c}
    T' \\
    N' \\
    B' 
  \end{array} = 
  \Big(\begin{array}{ccc}
    0 & \kappa & 0 \\
    -\kappa & 0 & \tau \\
    0 & \tau & 0
  \end{array} \Big)
  \begin{array}{c}
    T \\
    N \\
    B 
  \end{array}
\]
\begin{proposition*}
  Considere a la curva paramétrica regular $[g]$. Sean entonces $f ,\hat{f} \in
  [g]$ curvas regulares con $f:[a,b] \to \mathbb{R}^n$ y
  $\hat{f}:[\hat{a},\hat{b}] \to \mathbb{R}^n$. Considere entonces sus
  parametrizaciones naturales $F = f \circ s^{-1}$, $\hat{F} = \hat{f} \circ
  \hat{s}^{-1}$. Si $\hat{f} = f \circ \alpha$ con $\alpha$ una función de
  cambio de variable monótona creciente. Entonces $F(u)
  = \hat{F}(u) \quad \forall u \in s([a,b])$
\end{proposition*}
\begin{proof}
 Veamos que 
 \begin{align*}
   F &= f \circ s^{-1} \\
   \hat{F} &= \hat{f} \circ \hat{s}^{-1} = f \circ \alpha \circ \hat{s}^{-1}
 \end{align*}
 Así basta únicamente probar que $s^{-1} = \alpha \circ \hat{s}^{-1}$. Veamos que:
 \begin{align*}
    &\, &  s^{-1} &= \alpha \circ \hat{s}^{-1} \\
    &\Leftrightarrow &  s&= \hat{s} \circ \alpha^{-1} \\
    &\Leftrightarrow & s \circ \alpha \circ \hat{s}^{-1} (u) &= u 
 \end{align*}
Veamos que  
\begin{align*}
  (s \circ \alpha \circ \hat{s}^{-1})' &= (\alpha \circ \hat{s}^{-1})'(s' \circ
  \alpha \circ \hat{s}^{-1}) \\
                                       &= (\hat{s}^{-1})'(\alpha' \circ
                                       \hat{s}^{-1})\cdot (s' \circ \alpha \circ
                                       \hat{s}^{-1}) \\
                                       &= \frac{\alpha' \cdot ||f' \circ \alpha
                                       ||}{\hat{s}'} \circ \hat{s}^{-1}
\end{align*}
Veamos que $\alpha' >0$ puesto que $\alpha$ es monótona creciente. Además
\[
  \hat{s} = \int_a^t ||\hat{f}'|| = \int_a^t |\alpha'|\,||f' \circ \alpha||
\]
Así:
\[
  (s \circ \alpha \circ \hat{s}^{-1})' = 
  \frac{\alpha' \cdot ||f' \circ \alpha||}{|\alpha'| \, ||f' \circ \alpha||}
  \circ \hat{s}^{-1} = 1
\]
Con lo que tenemos que
\[
  s \circ \alpha \circ \hat{s}^{-1} (u) = u + C
\]
Y
\begin{align*}
  C &= s \circ \alpha \circ \hat{s}^{-1} (0) \\
    &= s \circ \alpha (\hat{a}) \\
    &= s (a) \\
    &= 0
\end{align*}
Con lo cual mostramos que $s^{-1} = \alpha \circ \hat{s}^{-1}$ o lo que es
equivalente:
$F = \hat{F}$
\end{proof}
\begin{definition}
  Sea $f: I \to \mathbb{R}^3$ la representación paramétrica natural de una curva
  paramétrica regular. Se dice que una curva regular $g$ es la envolvente de la
  traza de $f$ si la traza de $g$ intersecta a las tangentes de $f$ formando
  ángulos rectos 
\end{definition}
\begin{figure}[ht]
    \centering
    \incfig{envolvente}
    \caption{envolvente}
    \label{fig:envolvente}
\end{figure}
$g$ debe estar definida de la siguiente manera 
\[
  g(s) = f(s) +\lambda T(s)
\]
Con la condición $\langle g'(s), T(s) \rangle = 0$. Se tiene así
\begin{align*}
  0 = \langle g'(s), T(s) \rangle &= \langle f'(s) + \lambda'(s)T(s)+
  \lambda(s)T'(s) \rangle \\ 
                                  &= \langle'(s), T(s) \rangle +
                                  \lambda'(s)\langle T(s),T(s)\rangle + \lambda
                                  \langle T'(s), T(s) \rangle \\
                                  &= \langle f'(s) , f'(s) \rangle + \lambda'(s)\\
                                  &=  ||f'||^2 + \lambda' = 1 + \lambda'
\end{align*}
En consecuencia $\lambda(s) = -s + C$. y así se tiene que:
\[
  g(s) = f(s) + (C-s)T(s)
\]
Observe que no necesariamente $s$ es el parámetro natural para $g$.
También veamos que para cada valor de la constante $c$
\[
  \langle f'(s) + cT'(s) - s T'(s) - T(s) , T(s) \rangle = 0
\]

Se ha probado así la siguiente proposición
\begin{proposition}
  Dada una parametrización natural $f: I \to \mathbb{R}^3$ de clase $C^2$, la
  traza de $f$ tiene una infinidad de envolventes cada una dada por
  \[
    f(s) + (c-s)T(s)
  \]
  siendo $c$ una constante.
\end{proposition}

Ejemplo: considere la curva regular:
\begin{align*}
  f:[0, 2\pi ] \to \mathbb{R}^2 \\
  t \mapsto (\cos t, \sin t)
\end{align*}
Observe que $f$ es la parametrización natural de la clase de $f$. 
Así cada envolvente de la traza de $f$ es
\[
  g(t) = (\cos t - (c-t) \sin t, \sin t + (c-t) \cos t))
\]

\begin{figure}[ht]
    \centering
    \incfig{cacarol}
    \caption{cacarol}
    \label{fig:cacarol}
\end{figure}
Así se tiene que para $c= 0$ se tiene que $g(t)$ es un punto en la
circunferencia con centro en el origen y radio $[1 + t^2]^{\frac{1}{2}}$

Demuestre que si un segmento de un de longitud $2\pi$ tiene un extremo fijo al
punto $(1,0)$ y se extiende hacia abajo en forma paralela al eje $y$, entonces
su otro extremo describe a una envolvente de la circunferencia con centro en el
origen y radio 1 al envolver tal segmento a la circunferencia en sentido de las
manecillas del reloj.

\begin{definition}
  Dada una curva regular $f: I \to \mathbb{R}^3$, las curvas que tienen como
  envolvente a $f$ se denominan evolutas de $f$. 
\end{definition}
Se tiene así que las tangentes de una evoluta de $f$ son ortogonales a $f$.
Considere una curva regular $f:I \to \mathbb{R}^3$. Sea $t \in I$. Existe una
infinidad de rectas ortogonales a $f$ en el punto $f(t)$, todas aquellas en el
plano paralelo al determinado por los vectores $N(t)$ y $B(t)$. \\
Si $h: I \to \mathbb{R}^3$ es una evoluta de $f$, existen $\alpha: I \to
\mathbb{R}$, $\beta: I \to \mathbb{R}$ tales que:
\[
  \forall t \in I \quad h(t) = f(t) + \alpha(t) + \beta(t)B(t)
\]
y además $h'(t)$ debe ser paralelo al vector $\alpha(t) N (t) + \beta(t) B(t)$,
así debe cumplirse
\[
  h'(t) \times \alpha(t) N (t) + \beta(t) B(t)
\]
Se tiene entonces
\begin{align*}
  h' &= f'  + \alpha' N + \alpha N' + \beta' B + \beta B' \\
     &= s'T +  \alpha N + \alpha (- \kappa s'T - \tau s' B) +\beta'B +
     \beta(\tau s' N) \\
     &= s'(1-\k\alpha)T + (\alpha' + \beta \tau s')N + (\beta' - \alpha \tau
     s')B
\end{align*}
Se tiene así que :
\[
  0 = s'(1-\kappa \alpha) \alpha (T \times N) + s' (1-\kappa\alpha)\beta(T\times
  B) + [(\alpha' + \beta \tau s') \beta - (\beta- \alpha \tau s')\alpha (N
  \times B)]
\]
Como tenemos una combinación lineal de tres vectores linealmente independientes
iguales a cero, cada escalar es cero y en consecuencia:
\begin{align*}
  1- \kappa \alpha &= 0 \\
  \kappa = \rho \\
  [(\frac{\beta}{\alpha})^2 -1 ]\tau s' &=  \frac{\alpha\beta' -
  \alpha'\beta}{\alpha^2}\\
  \tau s' &= \frac{(\frac{\beta}{\alpha})^2}{(\frac{\beta}{\alpha})^2 + 1} \\
          &= (\arctan (\frac{\beta}{\alpha}))'  \\
  \beta &= \rho \tan\big[ \int \tau s'  + C \big]
\end{align*}
Se verifica así la siguiente proposición:
\begin{proposition}
  Dada una curva regular de clase $C^2$, de clase $C^2$ $f: I \to \mathbb{R}^3$,
  para la cual $\kappa(t) \neq 0$ entonces existe una evoluta de $f$ dada por:
  \[
    h = f + \alpha N + \beta B
  \]
  con $\alpha = \rho$ y  $\beta = \rho \tan \big[\int \tau s' + C\big]$
\end{proposition}
\begin{remark}
 Si $f$ es una curva plana $\tau = 0$. Así para $C = 0$, se tiene la evoluta $h
 = f + \rho N$. $h$ es entonces el lugar geométrico de los centros de curvatura.
\end{remark}
Considere la parábola definida por:
\begin{align*}
  f:I &\to \mathbb{R}^2  \\
  x &\mapsto (x,x^2)
\end{align*}
Se tiene entonces que:
\begin{align*}
  f'(x) &= (1,2x) \\
  || f' || &= (1+ 4x^2)
\end{align*}
Así:
\[
  T = \frac{f'}{||f'||} = \big( \frac{1}{\sqrt{1+1x^2}}, \frac{2x}{\sqrt{1+4x^2}}\big)
\]
Entonces 
\begin{align*}
  T' &= \frac{2}{(1 + 4x^2)^{3 /2}}  \\
  ||T'|| &= \frac{2}{1+4x^2} \\
  N &= \frac{T'}{||T'||} = \frac{(-2x,1)}{(1+4x^2)^{1 / 2}} \\
  \kappa &= \frac{||T'||}{||f'||} = \frac{2}{(1+4x^2)^{\frac{3}{2}}} \\
  \frac{N}{k} &= (-x - 4x^3 , \frac{1}{2} + 2x^2) \\
  h &= f + \frac{N}{\kappa} = (-4x^3, 3x^2 +\frac{1}{2})
\end{align*}

Si $y = y(u)$ define una función real de variable real cuya gráfica es la
evoluta de la parábola entonces se tendría:
\[
  y(-4x^3) = 3x^2 + \frac{1}{2}
\]
ó bien
\[
  y \circ g(x) = t(x)
\]
Siendo $g(x) = -4x^3$ y $t(x) = 3x^2 + \frac{1}{2}$

Así
\begin{align*}
  u &= g \circ g^{-1} \\
    &= -4 g^{-1} (u)^3 \\
  g^{-1}(u) &= (-\frac{u}{4})^{\frac{1}{3}} + \frac{1}{2}  
\end{align*}
Así la gráfica de $y$ es simérica respecto al eje de las ordenadas.
\begin{align*}
  y'(u)  &= \frac{u}{8}\big( \frac{u^2}{16} \big)^{-\frac{2}{3}} > 0 \mbox{ si } u>0\\
y''(u) &= - \frac{1}{24} \big(\frac{u^2}{16} \big)^{-\frac{2}{3}} < 0
\end{align*}
La ecuación de la evoluta 
\[
  y = 3 \big( \frac{x^2}{16} \big)^{\frac{1}{3}} + \frac{1}{2}
\]
Puede escribirse como:
\[
  (y - \frac{1}{2})^3 = 27 \frac{x^2}{16}
\]
o bien
\begin{equation}\label{evoluta:lambda}
  2(2y - 1)^3 = 27x^2
\end{equation}



\subsection{Parábola no homogénea}
\begin{figure}[ht]
    \centering
    \incfig{lamina}
    \caption{lamina}
    \label{fig:lamina}
\end{figure}
$(t,t^2)$ punto de contacto \\
$(a,b)$ centro de la lámina de la placa \\
\[
  y = 2xt - t ^2 \qquad \mbox{ ecuación de }L
\]
\[
  y = - \frac{1}{2t} (x -a) + b \qquad  \mbox{ ecuación de }M
\]
\[
  h = d((a,b),q) = \frac{b - 2ta + t^2}{(4t^2 +1)^{1 / 2}}
\]
\[
  V = mgh
\]
Así 
\[
  V(t) = mg \big[ \frac{b-2ta + t^2}{(4t^2 +1)^{1 / 2}} \big]
\]
Veamos que $V'(t) = 0$ si $(t,t^2)$ es un punto de equilibrio.
\[
  V'(t) = \frac{ mg u(t)}{(4t^2 + 1)^{3 / 2}}
\]
Siendo $u(t) = 2t^3 + t(1- 2b) - a$. \\
$(t,t^2)$ corresponde a un punto de equilibrio si y sólo si $u(t) = 0$, y además
$u$ pasa de negativo a positivo en $t$. \\
Observe que si $(t, t^2)$ es un punto de equilibrio $u(t) = 0$. Así $(a,b)$
pertenece a la recta perpendicular a $L$ que pasa por $(t, t^2)$; es decir,
$(t,t^2) \in M$.
Sea $S = \{ (t,a,b) \in \mathbb{R}^3 \, | \, 2 t^2 + t(1-2b) - a = 0 \, \}$.

Si se considera el plano $t=t_0$, $S$ lo intersecta en la recta $2 t_0^3 +
t_0(1-2b) -a = 0$. 
Sea $K$ la recta normal a la parábola $b = a^2$ en el plano $(a,b)$ en el punto
$(t_0,t_0^2)$. La ecuación de dicha recta es $b =  - \frac{1}{2t_0} a + t_0^2 +
\frac{1}{2}$. \\
Si la recta $K$ se traslada al plano paralelo al plano $ab$ que intersecta al
eje $t$ en $t_0$ lo que se tiene es precisamente la intersección de $S$ con tal
plano.
\begin{figure}[ht]
    \centering
    \incfig{superficie-s}
    \caption{superficie S}
    \label{fig:superficie-s}
\end{figure}
Considere un plano paralelo al plano $at$, digamos el plano $b = b_0$. Tal plano
intersecta a $S$ en $\{ (t, a, b_0) \, | \, 2t^3 + t(1-2b_0) -a = 0 \,\}$. Es
decir, en el conjunto definido por:
\begin{align*}
  a(t) &= 2t^3 + t(1-2b_0) \\
  a'(t) &= 6t^2 + c
\end{align*}
Así $a(t)$ tiene puntos críticos si y sólamente si $c \leq 0$, es decir, $b_0
\geq  \frac{1}{2}$. \\
\begin{figure}[ht]
    \centering
    \incfig{sabana}
    \caption{sabana}
    \label{fig:sabana}
\end{figure}

Para $(a,b)$ dado, la línea vertical a través de $(0,a,b)$. intersecta a $S$ en
un punto$(t,a,b)$ que es solución de la ecuación $u(t)=0$. El número de
soluciones puede ser 1,2 ó 3. Los valores de $t$ corresponden a puntos críticos
de $V$. Un mínimo de $V$ ocurre para valores de $t$ en los cuales $u(t)$ cambia
de negativo a positivo. \\
Sea $\gamma$ la región del plano $ab$ que consiste de los puntos $(a,b)$ para
los cuales hay exáctamente dos soluciones de $u$. 
Así si $t$ es la raíz doble:
\begin{align}
  2t^3 + t(1-2b) -a &= 0 \\ 
  6t^2 +(1-2b) &= 0 \\
  t &\neq 0 
\end{align}

Se tiene entonces que $2(2b -1)^3 = 27a^2$.

20-Septiembre-2019

\section{Superficies}
\begin{definition}
  Un conjunto $S \subset \mathbb{R}^3$ es una superficie si para cada punto $P
  \in S$ existen abiertos $U_P, V_P$ en $\mathbb{R}^2$ y $\mathbb{R}^3$
  respectivamente, así como un homeomorfismo: $f_P: U_P \to V_P \cap S$. Siendo
  $P \in V$. Si además $f_P$ es diferenciable y $\forall c \in U$, $Df_P(c)$ es
  inyectiva, entonces se dice que la superficie es regular. \\
  $f_P$ se denomina parametrización local de $S$ en $P$. Sea $A$ un conjunto de
  parametrizaciones locales de $S$; si para cada $P \in S$, existe en $A$ una
  parametrización local de $S$ en P entonces se dice que $A$ constituye un atlas
  para $S$, y a cada elemento de $A$ se le denomina carta.
\end{definition}
\begin{remark}
  En la definición de superficie regular, la condición relativa a que $D f_P(c)$
  sea inyectiva es equivalente a que las columnas de la matriz:
  \[
    \begin{bmatrix}
      D_1(f_P)_1 & D_2(f_P)_1 \\ 
      D_1(f_P)_2 & D_2(f_P)_2 \\ 
      D_1(f_P)_3 & D_2(f_P)_3 \\ 
    \end{bmatrix}
  \]
  Sean linealmente independientes.
\end{remark}
\begin{example}
  Sea $ S = \{(x,y,z) \, | \, x^2 + y^2 + z^2 = 1 \, \}$. Sea $P \in S$, $P =
  (P_1,P_2, P_3)$ Considere el caso en que $P_3 > 0$. Considere los conjuntos $U
  = \{ \, (x,y) \in \mathbb{R}^2 \, | \, x^2 + y^2 <1 \, \}$ y $V= \{ (x,y,z) \,
  | \, z > 0 \}$.  \\
  Considere la función:
  \begin{align*}
    f_P: U &\to V \cap S \\
    (x,y) &\mapsto (x,y, \sqrt{1-(x^2 + y^2)})
  \end{align*}
  $f_P$ es inyectiva, suprayectiva y contínua. Observe que $\f^{-1}_P: V \cap S
  \to U$ es contínua por ser la restricción de la proyección $ \Pi:  \mathbb{R}^3
  \to \mathbb{R}^2$ al conjunto $V \cap S$. \\
  $f_P$ es en consecuencia un homeomorfismo. $f_P$ es diferenciable ya que cada
  una de sus componentes lo es, y
  \[
    [Df_P(c)] = \begin{bmatrix}
      1 & 0 \\ 
      0 & 1 \\
      \frac{-c_1}{\sqrt{1-(c_1^2 +c_2^2)}} &
      \frac{-c_2}{\sqrt{1-(c_1^2 +c_2^2)}} \\
    \end{bmatrix}
  \]
que es inyectiva. 
La función anterior $f_P$ se denotará por $g_1$.

Sea $W = \{ (x,y,z) \, | , \mathbb{R}^3 \, | z <0 \}$. Considere a
\begin{align*}
  g_2: U &\to S \cap W$  \\
(x,y) &\mapsto (x, y , -\sqrt{1- (x^2 + y^2)}) \}
\end{align*}
$g_2$ es una parametrización local de $S$ en $a$ para cada $q \in S \cap W$. De
la misma forma se construyen $g_3, \ldots, g_6$ :
\begin{align*}
  g_3: U' &\to S \cap H \\
  (z,y) &\mapsto (\sqrt{1-(z^2 + y^2)}, y, z) \\
  g_4: U' &\to S \cap K \\
  (z,y) &\mapsto (-\sqrt{1-(z^2 + y^2)}, y, z) \\
  g_5: U'' &\to S \cap L \\
  (x,z) &\mapsto (x, \sqrt{1-(z^2 + x^2)}, z) \\
  g_6: U'' &\to S \cap M \\
  (x,z) &\mapsto (x, -\sqrt{1-(z^2 + x^2)}, z)
\end{align*}
Definiendo a los dominios de forma análoga.
Así se construye un átlas para la esfera. Se demuestra entonces que $S$ es una
superficie regular.
\end{example}
\begin{proposition}
  Sea $U$ abierto en $\mathbb{R}^2$ y sea $f: U \to \mathbb{R}$ una función
  diferenciable; entonces la gráfica de $f$ es una superficie regular

\end{proposition}
\begin{proof}
 Considere la función
 \begin{align*}
   g: U &\to \{ (x, y, f(x,y)) \, | \, (x,y) \in U \, \} \\
   (x,y) &\mapsto (x, y , f(x,y))
 \end{align*}
 $g$ es inyectiva, suprayectiva, derivable, y $g^{-1}$ es continua. Se tiene así
 que $g$ es un homeomorfismo. Y
 \[
   [Dg(c)] =%
   \begin{bmatrix}
     1 & 0 \\
     0 & 1 \\
     D_1f(c) & D_2f(c)
   \end{bmatrix}
 \]
Se ha probado que $g$ es una parametrización local de la gráfica de $f$ en
cualquier punto de ella.

\end{proof}
\begin{definition}
Sea $U$ un subconjunto abierto de $\mathbb{R}^3$. Considere una función $f: U
\to \mathbb{R}$ diferenciable. Se dice que $a \in f(U)$ es un punto singular de
$f$ si $Df(c)$ no es suprayectiva, en tal caso se dice que $f(c)$ es un valor
singular. Se dice que $a \in \mathbb{R}$ es un valor regular de $f$ si no es un
valor singular. 

\end{definition}
La siguiente proposición se basa en los planteamientos del teorema de la función
inversa. El teorema se encuentra en la mayor parte de los textos de la geometría
diferencial.

\begin{proposition}
  Sea $f: U \subset \mathbb{R}^3 \to \mathbb{R}$, siendo $U$ abierto, una
  función diferenciable. Sea $a \in f(u)$ un valor regular de $f$, entonces
  $f^{-1}(a)$ es una superficie regular.  
\end{proposition}
\begin{proof}
  Sea $S = f^{-1}(a)$ y sea $c \in S$. Como $f(c) = a$ y $a$ es un valor regular
  $c$ no es un punto singular. Por ello $Df(c)$ es suprayectiva, así
  \[
    [Df(c)] =%
    \begin{bmatrix}
      D_1f(c) & D_2f(c) & D_3f(c)
    \end{bmatrix}
  \]
  Con algún $D_if(c)$ distinto de 0. Asumiremos que $D_3f(c) \neq 0$; en caso
  necesario se renombrarían los ejes.

  Considere la función
  \begin{align*}
    F: U \subset \mathbb{R}^3 &\to \mathbb{R}^3 \\
    (x,y,z) &\mapsto (x,y, f(x,y,z))
  \end{align*}
  $F$ es claramente diferenciable y
  \[
    [DF(c)] =%
    \begin{bmatrix}
      1 & 0 & 0 \\ 
      0 & 1 & 0 \\ 
      D_1f(c) & D_2f(c) & D_3f(c) 
    \end{bmatrix}
  \]
  Observe que $\det [DF(c)] \neq 0$. Por el teorema de la función inversa
  existen abiertos, $V$, $W$ en $\mathbb{R}^3$ tales que $c \in V$ con $F(c) \in
  W$. $F: V \to W$, es una biyección diferenciable cuya inversa es
  diferenciable. 
Sean $g_1, g_2, g_3$ las componentes de $F^{-1}$, entonces cada una de ella es
diferenciable.

Sea 
\begin{align*}
  \Pi: \mathbb{R}^3 &\to \mathbb{R}^2 \\
  (x,y,z) &\mapsto (x,y)
\end{align*}
Observe que $\Pi(V) = \Pi(W)$.

Considere la función 
\begin{align*}
  h: \Pi(W) &\to \mathbb{R} \\
  (x,y) &\mapsto g_3(x,y,a)
\end{align*}
Veamos que si $(x,y) \in \Pi(W)$ existe $z' \in \mathbb{R}$ tal que $(x,y,z')
\in W$. Además $F(c_1,c_2,a) \in W$. Se puede asumir que $W$ es un producto
cartesiano de intervalos abiertos. Lo anterior justifica que $(x, y, a) \in W$. 

Sea
\begin{align*}
  k: \mathbb{R}^2 &\to \mathbb{R}^3 \\
  (x,y) &\mapsto (x,y,a)
\end{align*}
Veamos que $k$ es diferenciable. Se tiene que $h = g_3 \circ k|_{\Pi(W)}$. Se
tiene así que $h$ es diferenciable. Veamos que
\[
  F(V \cap S) = F(G_h) = \{ (x, y, a) \in W \} 
\]
Por lo cual $G_h = V \cap S$. Y anteriormente se mostró que la gráfica $G_h$ es
una superficie regular.


\end{proof}
\begin{proposition}
  Sean $S$ una superficie regular, $P \in S$, $f: U \subset \mathbb{R}^2 \to S$
  y $g:V \subset \mathbb{R}^2 \to S$ dos parametrizaciones locales de $S$ en $P$
  y $W = f(U) \cap g(V)$. La función $h = f^{-1} \circ g : g^{-1}(W) \to
  \f^{-1}(W)$ es un difeomorfismo.
\end{proposition}
\begin{proof}
  Sea $q = h(r)$, entonces $q \in f^{-1}(W) \subset U$. Como $f$ es
  parametrización local el rango de la  matriz $[Df(q)]$ es 2.
  En consecuencia se puede asumir que
  \[
     \det \begin{bmatrix} D_1f_1(q) & D_2f_1(q) \\
     D_1f_2(q) & D_2f_2(q) \end{bmatrix} \neq 0
  \]
Considere la función
\begin{align*}
  F: U \times \mathbb{R} &\to \mathbb{R}^3 \\
  (x,y,z) &\mapsto (f_1(x,y), f_2(x,y), f_3(x,y) + z)
\end{align*}
$F$ es diferenciable por ser cada una de sus componentes diferenciable.
Observe que $F(U \times \{0\}) = f(U)$
\[
  [DF(q)] = \begin{bmatrix} D_1f_1(q) & D_2f_1(q) & 0 \\
 D_1f_2(q) & D_2f_2(q) & 0 \\
 D_1f_3(q) & D_2f_3(q) & 1 
  \end{bmatrix} 
\]
Cuyo determinante es distinto de 0.
En consecuencia, del teorema de la función inversa existen abiertos $L$,$M$ en
$\mathbb{R}^3$ tales que $q \in L$ con $F(q) = f(q) \in M$ y $F:L \to M$ es una
biyección diferenciable cuya inversa es también diferenciable.
\begin{align*}
  g: V &\to g(V) \\
  r \in g^{-1}(W) &\subset V \\
  q = h(r) &= f^{-1} \circ g(r) \\
  \therefore f(q) &= g(r) \\
  \therefore g(r) \in M
\end{align*}
Como $M$ es abierto en $\mathbb{R}^3$ y $g(r) \in M$. Siendo $g$ continua,
existe un abierto $\hat{N}$ en $\mathbb{R}^2$ tal que $(\hat{N} \cap V)$
\end{proof}
\begin{definition}
 Sean $S$ una superficie regular, $P \in S$. Dada una parametrización local de
 $S$ en $P$
 \[
   f:U\subset \mathbb{R}^2 \to V \cap S
 \]
 $V \cap S$ se denomina vecindad coordenada de $P$.


Sea $(x,y) = f^{-1}(P) \in U$, $(x,y)$ se denominan las coordenadas de $P$ bajo
$f$. En general si $q \in V \cap S$ y $(a,b) = f^{-1}(q)$, a la pareja $(a,b)$
se le denominan las coordenadas de $q$ bajo $f$.
\end{definition}
\begin{remark}
  Si $f,g$ son dos parametrizaciones locales de $S$ en $P$ y $(x,y)$,
  $(\hat{x},\hat{y})$ son las coordenadas de $P$ bajo $f$ y $g$ respectivamente,
  se tiene que $h = g^{-1} \circ g$ es un difeomorfismo que transforma las
  coordenadas de $P$ bajo $g$ a las coordenadas de $P$ bajo $f$.
\end{remark}
\begin{definition}
 Sean $S$ una superficie regular, $V$ un abierto en $S$
 \[
   \phi: V \subset S \to \mathbb{R}^n
 \]
 Se dice que $\phi$ es diferenciable en $P \in V$ si dada una parametrización
 local de $S$ en $P$.
 \[
   g:U \subset \mathbb{R}^2 \to S
 \]
 tal que $g(U) \subset V$, se tiene que:
 \[
   \phi \circ g: U \subset \mathbb{R}^2 \to \mathbb{R}^n
 \]
 es diferenciable en $g^{-1}(P)$
\end{definition}
\begin{remark}
 La definición anterior no depende de la parametrización 

 Suponga que $\hat{g}$ es otra parametrización de $S$ en $P$, sea $h = g^{-1}
 \circ \hat{g}$. Así $\phi \circ \hat{g} = \phi \circ g \circ h$.
 Se tiene entonces que $\phi \circ \hat{g}$ es diferenciable en
 $\hat{g}^{-1}(P)$.

\end{remark}
\begin{definition}
 Sean $S_1$ y $S_2$ superficies regulares,
 \[
   \phi: V \subset S_1 \to S_2
 \]
 con $V$ abierto en $S_1$. $\phi$ se dice diferenciable en $P \in V$ si dadas
 parametrizaciones locales $g_1$,$g_2$ de $S_1$,$S_2$ en $P$ y $\phi(P)$
 respectivamente, cada una definida en $U_1$, $U_2$ tales que $g_1(U_1) \subset
 V$ y $\phi(g_1(U_1)) \subset g_2(U_2)$, se tiene que la función
 $g_2^{-1} \circ g \circ g_1$ es diferenciable en $g_1^{-1}(P)$. 

\end{definition}
\begin{definition}
 Dos superficies regulares se $S_1$, $S_2$ se dicen difeomorfas si existe una
 biyección $\phi:S_1 \to S_2$ diferenciable cuya inversa es diferenciable.

\end{definition}

\begin{proposition}
  Sea $S$ una superficie regular, $P \in S$. $f: U \subset \mathbb{R}^2 \to S$
  una parametrización local de $S$ en $P$, entonces $f^{-1}:f(U) \t \mathbb{R}^2$  
  Sea $g: V \subset \mathbb{R}^2$ es una función diferenciable

\end{proposition}
\begin{proof}
  Sea $P \in f(U)$

  Sea $g: V \subset \mathbb{R}^2 \to S$ una parametrización local se $S$ en $P$.
  Sea $h = f^{-1} \circ g: g^{-1}(W) \to f^{-1}(W)$ con $W= f(U) \cap g(V)$ $h$
  es en particular difrerenciable en $g^{-1}(P)$, así $f^{-1}$ diferencaible en
  $P$.

\end{proof}
La proposición implica que siendo $S$ una superficie regular y $f$ una
parametrización local de $S$ en $P$. $f: U \to f(U)$ es un difeomorfismo. Ésto
muestra que toda superficie regular es localmente difeomorfa al plano.

\begin{problem}
  Sean $V \subset \mathbb{R}^3$ abierto, $S$ una superficie regular:
  $f:V \to \mathbb{R}^n$ una función diferenciable.
  Si $S \subset V$ demuestre que $f|_S$ es una función diferenciable.

\end{problem}
\begin{problem}
  Demuestre que toda esfera es difeomorfa a toda elipsoide

\end{problem}
